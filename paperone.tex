\documentclass{article}
\usepackage[utf8]{inputenc}

\title{Computer Graphics on Phone is On Par with PC Desktop}
\author{Kevin Le}
\date{October 21 2020}

\begin{document}

\maketitle

\begin{abstract}
Mobile devices continue to improve and become more powerful every year. Despite them having less hardware power than desktop computers and being portable limiting optimal performances, mobile devices can handle computer graphic tasks well. Most if not all mobile devices nowadays contains sensors, a camera, and a GPU which when taken into account makes mobile devices more useful than ever. The advancement of tablets and phones informs us that they can handle computer graphics task on par with desktop computers. Even though mobile devices still lack in performances when against desktop PC's, they can handle tough computer graphic tasks now such as 3D volume rendering and ray tracing which was considered to be too much for portable devices to handle not too long ago. This paper will discuss tasks mobile devices can handle that makes it on par with desktop computers when it comes to handling 3D computer graphics tasks. 
\end{abstract}

\section{Introduction}
Technology has continued to grow and improve exponentially for many years. Mobile devices are no exception with large companies like Apple and Microsoft continuing to upgrade their mobile devices every year to make them more powerful than the last. Besides the constant improvements on things like better battery life, screen, and human to machine interactions, there are two key pieces of hardware that really matter when factoring in the mobile devices ability to complete 3D graphical tasks, the CPU and GPU. These hardware pieces are important since they manage graphic performances. Their improvements in mobile devices has made them become more useful and stronger than ever. One graphical rendering task that has been known to put pressure on desktop computers is ray tracing. This rendering technique is used to produce realistic lighting and shading effects in 3d models and games. If the GPU or CPU is not strong enough to handle rendering tasks such as ray tracing, it will hurt the devices performances or worse harm the device itself due to overheating. It was believed that mobile devices was not strong enough to handle intense 3D computer graphic tasks such as ray tracing due to their battery limitation and weaker hardware components. That however was proven not true as some mobile devices can indeed handle large rendering tasks such as ray tracing and 3D volume rendering. Coupled with the ability to be able to create 3D model characters and objects, it seems that mobile devices nowadays is on par with desktop computers when dealing with 3D graphical tasks.

\section{Ray Tracing and 3D Volume Rendering}
  One group of researchers was able to do ray tracing on mobile that is par with desktop computers. This was because the group focused on two key features, the hardware accelerator and programmable shaders \cite{10.1145/2543651.2543670}.
  
\section{Other 3D Graphical Tasks}
  A team was able to produce an app called Bot3D that allows users to produce 3D animated characters on mobile devices with little no problem. The best part is they were able to make it work just like how it would on desktop PC \cite{10.1145/3132787.3132807}.
  
\cite{7579401}
\cite{7111373}
\cite{7980370}
\cite{8047282}
\cite{7030191}

\cite{10.1145/3365610.3368420}
\cite{10.1145/2503512.2503521}

\section{Discussion}
  The idea that mobile devices can handle 3D volume rendering and ray tracing without much problem this generation makes me believe that mobile devices will only continue to become more powerful. Despite the constant upgrade in hardware for mobile devices every year, I doubt there will ever be a time where mobile devices will out perform PC. This is again due to the fact that mobile devices are run by a battery which really limits it from performing more optimally than PC when executing intense graphical tasks. Mobile devices will continue to constantly improve but will sadly never outshine desktop computers. One possible idea researchers can do in the distant future to make mobile devices perform better in 3D graphical tasks is to drastically improve the battery source or create a very powerful GPU or CPU that would not overheat too much. I believe that day will eventually come, but it will take along while before that happens. 
\section{Conclusion}
 In the end, mobile devices were able to handle intense 3D graphical tasks that are usually do able only on desktop PC. 
\bibliographystyle{plain}
\bibliography{bibliography.bib}

\end{document}
