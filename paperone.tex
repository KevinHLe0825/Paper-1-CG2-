\documentclass{article}
\usepackage[utf8]{inputenc}

\title{Computer Graphics on Phone is On Par with PC Desktop}
\author{Kevin Le}
\date{October 21 2020}

\begin{document}

\maketitle

\begin{abstract}
   Mobile devices continue to improve and become more powerful every year. Despite them having less hardware power than desktop computers and being portable limiting optimal performances, mobile devices can handle computer graphic tasks. Most if not all mobile devices nowadays contains sensors, a camera, and a GPU which when taken into account makes mobile devices more useful than ever. The advancement of tablets and phones informs us that they can handle computer graphics task on par with desktop computers. Tasks such as ray tracing, 3D volume rendering, and 3D character creations was considered too much for portable devices to deal with not too long ago. This paper will go over the improvements of mobile devices and the graphic tasks mobile devices can do that makes it on par with desktop computers when dealing with computer graphics tasks. 
\end{abstract}

\section{Introduction}
   Technology has continued to grow and improve exponentially for many years. Mobile devices are no exception with large companies like Apple and Microsoft continuing to upgrade their mobile devices every year to make them more powerful than the last. Besides the constant improvements on things like better battery life, screen, and human to machine interactions, there are two other pieces of hardware that primarily affect the performances and outcome of graphical tasks on mobile devices, the CPU and GPU. These two hardware pieces are important since they handle many graphical operations such as shaders, ray-tracing, and rendering. With improvements to the CPU and GPU in mobile devices, they become stronger and can withstand tougher 3D graphic tasks. One graphic task that been known to put pressure on desktop computers is ray tracing. This rendering technique is used to produce realistic lighting and shading effects in 3d models and games. If the GPU or CPU is not strong enough to handle rendering tasks such as ray tracing, it will hurt the devices performances or even worse, harm the device itself due to overheating. Due to battery limitations and weaker hardware components in mobile devices, it was hard to believe that mobile devices were capable of running intense graphic tasks. However, because of how much mobile devices have progressed, society today has reached a point where mobile devices are able to deal rough 3D graphic tasks now such as large 3D volume rendering and ray tracing. Mobile devices this generation are also able to deal with graphic tasks that are usually done on desktop computers as well. Even though portable devices struggle to maintain good performances overall for tough 3D graphical tasks, just being able to do the tasks in general like desktop computers makes mobile devices on par with desktop computers.

\section{3D Volume Rendering and Ray Tracing}
  Ray tracing has become a widely known graphic technique that is used in many 3D video games, movies, and shows. It is one of the rendering techniques in computer graphics which enhances 3D graphics by creating realistic effects with lighting. The work behind ray tracing is known to be intense and can put pressure on hardware such as the CPU and GPU on desktop computers. Because of this, many researchers question if it is possible to implement ray tracing on mobile devices since they have battery limitation and weaker hardware compared to PC. Although it seems impossible, many computer graphic users still strive to discover if there are any possible techniques on mobile devices that can make ray tracing even slightly possible. One team managed to implement mobile ray tracing by creating a new mobile GPU architecture called SGRT which is a Samsung re-configurable GPU and by using the hardware accelerator and programmable shaders combined \cite{10.1145/2543651.2543670}. With SGRT and their hardware engine which accelerates traversal and interaction operations (T&I unit), the team was able to produce ray tracing and even improve performances with a new parallel pipeline. The teams system design goes as follows, a Mobile Computing Platform was used to integrate multi-core CPUs and many-core GPUs. Because T&I operations consume most resources, a dedicated hardware unit was enforced to fix this issue. A programmable shader called SRP which stands for Samsung Re-configurable Processor is used by the team to support ray and shading algorithms. Hardware acceleration is then implemented to simplify the hardware structures. Tree build operations are then assigned to multi-core CPUs along with a single ray tracing architecture to complete the ray tracing process on mobile devices. With the teams SGRT system, they were able to produce promising results. A decent 30 frames per second was produced along with the resolution being in full HD. Other desktop computers in the past were compared with the SGRT system as well with their resolution also being close to HD with a consistency of 30 frames per second as well. This results demonstrated that ray tracing is possible on mobile devices and that if done correctly can be on par with some of the past desktop computers.
  
   Another graphic technique that is known to cause trouble on mobile devices is large 3D volume rendering. Big volume data sets require a significant amount of computing power which is limited to desktop usually. Mobile has two main reasons for not being as strong as desktops, the first being overheating and resource poverty due to batteries. The second being network dependency in which mobile devices depend on wireless cellular network which limits memory. Despite these limitation and risks, a team behind 3D volume rendering managed to find a technique that successfully allows mobile devices to handle large renderings of 3D shapes. The technique consist of using one of the following three client methods and then doing 2D texture slicing \cite{7111373}. The first client is called thin client which moves the entire volume rendering task to a dedicated remote render server through wifi. This client was used before when mobile devices lacked GPU and needed to get rid of workload. The second client method called balanced attempts to do the 3D volume rendering between the server and the device itself. This method is used so that not everything is relied on the remote server alone. The third last client is called fat client which relies volume rendering solely on the client side with no servers. This client method was primarily used on mobile devices with strong hardware. Out of the three client methods, thin client ended up being the best one for mobile devices since it can handle volume management strategies such as 2D texture stacking and provides the best end result out of the other two clients. 2D Texture slicing is the idea of rendering a stack of 2D slices inside a volume. By doing his method and applying thin client, mobile devices are able render 3D volume data sets without sacrificing visual quality. The idea that mobile devices can render large 3D volume data sets without sacrificing quality similarly to desktop computers further support the idea that mobile devices are on par with desktop computers when it comes to dealing with computer graphic tasks.  
  
\section{Desktop to Mobile: 3D Tasks}
     
    Graphical tasks are primarily done on desktop computers because it contains powerful hardware pieces that get graphical job tasks done fast and efficiently. However, technology has evolved to the point where portability has become an important deciding factor for many tech companies. Technology businesses nowadays focus on creating mobile devices that can handle most if not all desktop computer tasks on the go. Graphical tasks are no exception, many researchers and app creators attempt to create or transfer graphical methods over to mobile devices. One research team produced an an app called Bot3D that allows users to produce 3D character animations on mobile devices \cite{10.1145/3132787.3132807}. The application was on desktop computers exclusively, but after the team discovered that some mobile devices were as complex as desktops, they decided to work on the app on portable devices. The Bot3D mobile app allowed users do handle-controlled 3D rotation, environmental settings, timeline, and key frames which are all used in 3D computer graphic animations. Bot3D also allowed joint and kinematic movements to 3D characters which according to Debuchi was designed to be performed on PC only. However, users on the mobile side can now use modules like pose, hand, face, and scenes which used to be exclusive to PC. Old users of Bot3D can now work and do character animations on their mobile devices and get the same results as they would get on desktop computers. Debuchi does mention that the Bot3D mobile app is limited to certain mobile devices, mainly those that have the same hardware power that can run the desktop version of Bot3D. Despite this limitation however, mobile devices in general were still able to run Bot3D and execute everything on par with the desktop version. 
  
    Interaction techniques between desktop devices to mobile devices were highly researched as well.  A group researched a way for users to interact with large 3D data sets on mobile devices when they are moved over from desktop computers. The solution is to render a new volume similar to the one on desktop but keep it smaller to increase frames. Downsizing is not good in the long term since it provides less accurate data and can cause severe visual artifacts data to be lost. Hence instead of downsizing, a method was created to split the large volumes into parts, sub volumes \cite{7579401}. Sub volumes zooms in the image and keeps the same clear detailed resolution. Since the image is split into parts, grid points were created to hold information about what to load and helps the user jeep track of where they are. By doing sub volume and using techniques such as grid points and bread slice view, researchers were able to render 3D data sets on mobile devices and have the frames be decent and the resolution be the same and detailed. Although the mobile devices were not able to take in the large 3D data set in one sitting like desktop computers and requires a separate method for it, they were still able to successfully take in large 3D data sets. With the sub volume method implemented to mobile devices, users are able to experience the interaction of large 3D data sets similarly to desktop computers. 

\section{Discussion}
   Mobile devices will only continue to become more powerful with the constant improvements they receive from tech companies. Despite this however, there will most likely never be a time where mobile devices will surpass or out perform desktop computers. As mentioned previously, this is primarily due to two key factors, mobile devices running on battery and it running on wireless network. Other users state another possible reason why mobile devices will never surpass desktop, the architectures of mobile GPU differ from desktops GPU. Mobile GPUs is limit in size which causes resource constraints and faster power consumption since it can overheat easier when handling intense graphic tasks \cite{10.1145/2503512.2503521}. Despite mobile devices being more limited than desktop, researchers will always continue to find a method or technique to make them execute a task that can equally be done on PC. 
   
   As mentioned earlier, portability has become an important factor in tech industries now. This is because mobile devices are much more convenient and can be used anywhere anytime. Since a majority of mobile devices contains a sensor, camera, and GPU, people find it more easier to complete graphical tasks nowadays. Some users go even further to argue that inbuilt sensors in mobile devices such as accelerometers and camera flash can extend computer graphics technology \cite{7030191}. Mobile devices have also evolved to the point where it can handle AR interactions and become a virtual reality headset itself. Virtual and augmented realities have become an important part of computer graphics lately since they both work and contain large sets of 3D graphics. It has gone to the point where researchers are finding ways to migrate tasks from desktops and mobile devices to wearable AR and virtual reality. One team behind drone work experimented to see if people prefer doing drone missions using desktop user interface or wearable AR user interface \cite{10.1145/3365610.3368420}. The experiment showed that users preferred wearable AR over desktop UI because it makes them become more engaged with the virtual environment. Since mobile devices can become a wearable virtual reality or AR headset nowadays, people can now perform graphical tasks like drone missions on the go. This shows that despite hardware limitations against desktop computers, users still prefer mobile devices due to portability reasons. Portability has become one of the biggest factor for users when it comes to technology. Even though mobile devices are weaker than PCs hardware wise, it can still get tasks done equally to its predecessor making them the better choice in most peoples eyes.
   
\section{Conclusion}
   Computer graphic tasks such as ray tracing and 3D volume rendering can be done on mobile devices. Limitations or not, researchers will always find some type of technique or way to handle graphical tasks on mobile devices. Whether it be creating a new GPU architecture or having a server do all the workload, there will be a solution. Even though the methods require a longer process and can take time, it still can execute the graphical tasks accordingly. As long as mobile devices can find a way to handle 3D graphic tasks that are normally done on PC, it is proven to be on par with desktop computers.
   
\bibliographystyle{plain}
\bibliography{bibliography.bib}

\end{document}
